\documentclass[]{article}
\usepackage{lmodern}
\usepackage{amssymb,amsmath}
\usepackage{ifxetex,ifluatex}
\usepackage{fixltx2e} % provides \textsubscript
\ifnum 0\ifxetex 1\fi\ifluatex 1\fi=0 % if pdftex
  \usepackage[T1]{fontenc}
  \usepackage[utf8]{inputenc}
\else % if luatex or xelatex
  \ifxetex
    \usepackage{mathspec}
  \else
    \usepackage{fontspec}
  \fi
  \defaultfontfeatures{Ligatures=TeX,Scale=MatchLowercase}
\fi
% use upquote if available, for straight quotes in verbatim environments
\IfFileExists{upquote.sty}{\usepackage{upquote}}{}
% use microtype if available
\IfFileExists{microtype.sty}{%
\usepackage{microtype}
\UseMicrotypeSet[protrusion]{basicmath} % disable protrusion for tt fonts
}{}
\usepackage[margin=1in]{geometry}
\usepackage{hyperref}
\hypersetup{unicode=true,
            pdftitle={02-similitud-minihashin-tarea},
            pdfauthor={Soledad\_Perez},
            pdfborder={0 0 0},
            breaklinks=true}
\urlstyle{same}  % don't use monospace font for urls
\usepackage{color}
\usepackage{fancyvrb}
\newcommand{\VerbBar}{|}
\newcommand{\VERB}{\Verb[commandchars=\\\{\}]}
\DefineVerbatimEnvironment{Highlighting}{Verbatim}{commandchars=\\\{\}}
% Add ',fontsize=\small' for more characters per line
\usepackage{framed}
\definecolor{shadecolor}{RGB}{248,248,248}
\newenvironment{Shaded}{\begin{snugshade}}{\end{snugshade}}
\newcommand{\KeywordTok}[1]{\textcolor[rgb]{0.13,0.29,0.53}{\textbf{#1}}}
\newcommand{\DataTypeTok}[1]{\textcolor[rgb]{0.13,0.29,0.53}{#1}}
\newcommand{\DecValTok}[1]{\textcolor[rgb]{0.00,0.00,0.81}{#1}}
\newcommand{\BaseNTok}[1]{\textcolor[rgb]{0.00,0.00,0.81}{#1}}
\newcommand{\FloatTok}[1]{\textcolor[rgb]{0.00,0.00,0.81}{#1}}
\newcommand{\ConstantTok}[1]{\textcolor[rgb]{0.00,0.00,0.00}{#1}}
\newcommand{\CharTok}[1]{\textcolor[rgb]{0.31,0.60,0.02}{#1}}
\newcommand{\SpecialCharTok}[1]{\textcolor[rgb]{0.00,0.00,0.00}{#1}}
\newcommand{\StringTok}[1]{\textcolor[rgb]{0.31,0.60,0.02}{#1}}
\newcommand{\VerbatimStringTok}[1]{\textcolor[rgb]{0.31,0.60,0.02}{#1}}
\newcommand{\SpecialStringTok}[1]{\textcolor[rgb]{0.31,0.60,0.02}{#1}}
\newcommand{\ImportTok}[1]{#1}
\newcommand{\CommentTok}[1]{\textcolor[rgb]{0.56,0.35,0.01}{\textit{#1}}}
\newcommand{\DocumentationTok}[1]{\textcolor[rgb]{0.56,0.35,0.01}{\textbf{\textit{#1}}}}
\newcommand{\AnnotationTok}[1]{\textcolor[rgb]{0.56,0.35,0.01}{\textbf{\textit{#1}}}}
\newcommand{\CommentVarTok}[1]{\textcolor[rgb]{0.56,0.35,0.01}{\textbf{\textit{#1}}}}
\newcommand{\OtherTok}[1]{\textcolor[rgb]{0.56,0.35,0.01}{#1}}
\newcommand{\FunctionTok}[1]{\textcolor[rgb]{0.00,0.00,0.00}{#1}}
\newcommand{\VariableTok}[1]{\textcolor[rgb]{0.00,0.00,0.00}{#1}}
\newcommand{\ControlFlowTok}[1]{\textcolor[rgb]{0.13,0.29,0.53}{\textbf{#1}}}
\newcommand{\OperatorTok}[1]{\textcolor[rgb]{0.81,0.36,0.00}{\textbf{#1}}}
\newcommand{\BuiltInTok}[1]{#1}
\newcommand{\ExtensionTok}[1]{#1}
\newcommand{\PreprocessorTok}[1]{\textcolor[rgb]{0.56,0.35,0.01}{\textit{#1}}}
\newcommand{\AttributeTok}[1]{\textcolor[rgb]{0.77,0.63,0.00}{#1}}
\newcommand{\RegionMarkerTok}[1]{#1}
\newcommand{\InformationTok}[1]{\textcolor[rgb]{0.56,0.35,0.01}{\textbf{\textit{#1}}}}
\newcommand{\WarningTok}[1]{\textcolor[rgb]{0.56,0.35,0.01}{\textbf{\textit{#1}}}}
\newcommand{\AlertTok}[1]{\textcolor[rgb]{0.94,0.16,0.16}{#1}}
\newcommand{\ErrorTok}[1]{\textcolor[rgb]{0.64,0.00,0.00}{\textbf{#1}}}
\newcommand{\NormalTok}[1]{#1}
\usepackage{graphicx,grffile}
\makeatletter
\def\maxwidth{\ifdim\Gin@nat@width>\linewidth\linewidth\else\Gin@nat@width\fi}
\def\maxheight{\ifdim\Gin@nat@height>\textheight\textheight\else\Gin@nat@height\fi}
\makeatother
% Scale images if necessary, so that they will not overflow the page
% margins by default, and it is still possible to overwrite the defaults
% using explicit options in \includegraphics[width, height, ...]{}
\setkeys{Gin}{width=\maxwidth,height=\maxheight,keepaspectratio}
\IfFileExists{parskip.sty}{%
\usepackage{parskip}
}{% else
\setlength{\parindent}{0pt}
\setlength{\parskip}{6pt plus 2pt minus 1pt}
}
\setlength{\emergencystretch}{3em}  % prevent overfull lines
\providecommand{\tightlist}{%
  \setlength{\itemsep}{0pt}\setlength{\parskip}{0pt}}
\setcounter{secnumdepth}{0}
% Redefines (sub)paragraphs to behave more like sections
\ifx\paragraph\undefined\else
\let\oldparagraph\paragraph
\renewcommand{\paragraph}[1]{\oldparagraph{#1}\mbox{}}
\fi
\ifx\subparagraph\undefined\else
\let\oldsubparagraph\subparagraph
\renewcommand{\subparagraph}[1]{\oldsubparagraph{#1}\mbox{}}
\fi

%%% Use protect on footnotes to avoid problems with footnotes in titles
\let\rmarkdownfootnote\footnote%
\def\footnote{\protect\rmarkdownfootnote}

%%% Change title format to be more compact
\usepackage{titling}

% Create subtitle command for use in maketitle
\newcommand{\subtitle}[1]{
  \posttitle{
    \begin{center}\large#1\end{center}
    }
}

\setlength{\droptitle}{-2em}

  \title{02-similitud-minihashin-tarea}
    \pretitle{\vspace{\droptitle}\centering\huge}
  \posttitle{\par}
    \author{Soledad\_Perez}
    \preauthor{\centering\large\emph}
  \postauthor{\par}
      \predate{\centering\large\emph}
  \postdate{\par}
    \date{10/2/2019}


\begin{document}
\maketitle

\subsection*{Tarea}\label{tarea}
\addcontentsline{toc}{subsection}{Tarea}

\begin{enumerate}
\def\labelenumi{\arabic{enumi}.}
\tightlist
\item
  Calcula la similitud de Jaccard de las cadenas ``Este es el ejemplo
  1'' y ``Este es el ejemplo 2'', usando tejas de tamaño \(3\).
\end{enumerate}

\begin{Shaded}
\begin{Highlighting}[]
\NormalTok{textos <-}\StringTok{ }\KeywordTok{character}\NormalTok{(}\DecValTok{2}\NormalTok{)}
\NormalTok{textos[}\DecValTok{1}\NormalTok{] <-}\StringTok{ 'Este es el ejemplo 1'}
\NormalTok{textos[}\DecValTok{2}\NormalTok{] <-}\StringTok{ 'Este es el ejemplo 2'}
\end{Highlighting}
\end{Shaded}

\begin{Shaded}
\begin{Highlighting}[]
\KeywordTok{library}\NormalTok{(tidyverse)}

\NormalTok{sim_jaccard <-}\StringTok{ }\ControlFlowTok{function}\NormalTok{(a, b)\{}
    \KeywordTok{length}\NormalTok{(}\KeywordTok{intersect}\NormalTok{(a, b)) }\OperatorTok{/}\StringTok{ }\KeywordTok{length}\NormalTok{(}\KeywordTok{union}\NormalTok{(a, b))}
\NormalTok{\}}

\NormalTok{shingle_chars <-}\StringTok{ }\ControlFlowTok{function}\NormalTok{(string, k, }\DataTypeTok{lowercase =} \OtherTok{FALSE}\NormalTok{)\{}
\NormalTok{    tokenizers}\OperatorTok{::}\KeywordTok{tokenize_character_shingles}\NormalTok{(string, }\DataTypeTok{n =}\NormalTok{ k, }\DataTypeTok{lowercase =} \OtherTok{FALSE}\NormalTok{,}
        \DataTypeTok{simplify =} \OtherTok{TRUE}\NormalTok{, }\DataTypeTok{strip_non_alphanum =} \OtherTok{FALSE}\NormalTok{)}
\NormalTok{\}}

\NormalTok{tejas_doc <-}\StringTok{ }\KeywordTok{map}\NormalTok{(textos, shingle_chars, }\DataTypeTok{k =} \DecValTok{6}\NormalTok{)}
\KeywordTok{sim_jaccard}\NormalTok{(tejas_doc[[}\DecValTok{1}\NormalTok{]], tejas_doc[[}\DecValTok{2}\NormalTok{]])}
\NormalTok{## [1] 0.875}
\NormalTok{tejas_doc[[}\DecValTok{1}\NormalTok{]]}
\NormalTok{##  [1] "Este e" "ste es" "te es " "e es e" " es el" "es el " "s el e"}
\NormalTok{##  [8] " el ej" "el eje" "l ejem" " ejemp" "ejempl" "jemplo" "emplo "}
\NormalTok{## [15] "mplo 1"}
\NormalTok{tejas_doc[[}\DecValTok{2}\NormalTok{]]}
\NormalTok{##  [1] "Este e" "ste es" "te es " "e es e" " es el" "es el " "s el e"}
\NormalTok{##  [8] " el ej" "el eje" "l ejem" " ejemp" "ejempl" "jemplo" "emplo "}
\NormalTok{## [15] "mplo 2"}
\end{Highlighting}
\end{Shaded}

\begin{enumerate}
\def\labelenumi{\arabic{enumi}.}
\setcounter{enumi}{1}
\tightlist
\item
  (Ejercicio de {[}@mmd{]}) Considera la siguiente matriz de
  tejas-documentos:
\end{enumerate}

\begin{Shaded}
\begin{Highlighting}[]
\NormalTok{mat <-}\StringTok{ }\KeywordTok{matrix}\NormalTok{(}\KeywordTok{c}\NormalTok{(}\DecValTok{0}\NormalTok{,}\DecValTok{1}\NormalTok{,}\DecValTok{0}\NormalTok{,}\DecValTok{1}\NormalTok{,}\DecValTok{0}\NormalTok{,}\DecValTok{1}\NormalTok{,}\DecValTok{0}\NormalTok{,}\DecValTok{0}\NormalTok{,}\DecValTok{1}\NormalTok{,}\DecValTok{0}\NormalTok{,}\DecValTok{0}\NormalTok{,}\DecValTok{1}\NormalTok{,}\DecValTok{0}\NormalTok{,}\DecValTok{0}\NormalTok{,}\DecValTok{1}\NormalTok{,}\DecValTok{0}\NormalTok{,}\DecValTok{0}\NormalTok{,}\DecValTok{0}\NormalTok{,}\DecValTok{1}\NormalTok{,}\DecValTok{1}\NormalTok{,}\DecValTok{1}\NormalTok{,}\DecValTok{0}\NormalTok{,}\DecValTok{0}\NormalTok{,}\DecValTok{0}\NormalTok{),}
              \DataTypeTok{nrow =} \DecValTok{6}\NormalTok{, }\DataTypeTok{byrow =} \OtherTok{TRUE}\NormalTok{)}
\KeywordTok{colnames}\NormalTok{(mat) <-}\StringTok{ }\KeywordTok{c}\NormalTok{(}\StringTok{'d_1'}\NormalTok{,}\StringTok{'d_2'}\NormalTok{,}\StringTok{'d_3'}\NormalTok{,}\StringTok{'d_4'}\NormalTok{)}
\KeywordTok{rownames}\NormalTok{(mat) <-}\StringTok{ }\KeywordTok{c}\NormalTok{(}\DecValTok{0}\NormalTok{,}\DecValTok{1}\NormalTok{,}\DecValTok{2}\NormalTok{,}\DecValTok{3}\NormalTok{,}\DecValTok{4}\NormalTok{,}\DecValTok{5}\NormalTok{)}
\NormalTok{mat}
\end{Highlighting}
\end{Shaded}

\begin{verbatim}
##   d_1 d_2 d_3 d_4
## 0   0   1   0   1
## 1   0   1   0   0
## 2   1   0   0   1
## 3   0   0   1   0
## 4   0   0   1   1
## 5   1   0   0   0
\end{verbatim}

\begin{itemize}
\tightlist
\item
  Sin permutar esta matriz, calcula la matriz de firmas minhash usando
  las siguientes funciones hash: \(h_1(x) = 2x+1\mod 6\),
  \(h_2(x) = 3x+2\mod 6\), \(h_3(x)=5x+2\mod 6\). Recuerda que
  \(a\mod 6\) es el residuo que se obtiene al dividir a entre \(6\), por
  ejemplo \(14\mod 6 = 2\), y usa la numeración de renglones comenzando
  en \(0\).
\item
  Compara tu resultado usando el algoritmo por renglón que vimos en
  clase, y usando el algoritmo por columna (el mínimo hash de los
  números de renglón que tienen un \(1\)).
\item
  ¿Cuál de estas funciones hash son verdaderas permutaciones?
\item
  ¿Qué tan cerca están las similitudes de Jaccard estimadas por minhash
  de las verdaderas similitudes?
\end{itemize}

\begin{enumerate}
\def\labelenumi{\arabic{enumi}.}
\setcounter{enumi}{2}
\tightlist
\item
  Funciones hash. Como vimos en clase, podemos directamente hacer hash
  de las tejas (que son cadenas de texto), en lugar de usar hashes de
  números enteros (número de renglón). Para lo siguiente, puedes usar la
  función \emph{hash\_string} del paquete \emph{textreuse} (o usar la
  función \emph{pyhash.murmur3\_32} de la librería \emph{pyhash}):
\end{enumerate}

\begin{itemize}
\tightlist
\item
  Calcula valores hash de algunas cadenas como `a', `Este es el ejemplo
  1', `Este es el ejemplo 2'.
\item
  Calcula los valores hash para las tejas de tamaño \(3\) de `Este es el
  ejemplo 1'. ¿Cuántos valores obtienes?
\item
  Usa los números del inciso anterior para calcular el valor minhash del
  texto anterior.
\item
  Repite para la cadena `Este es otro ejemplo.', y usa este par de
  minhashes para estimar la similitud de Jaccard (en general usamos más
  funciones minhash para tener una buena estimación, no solo una!).
\item
  Repite los pasos anteriores para \(10\) funciones minhash (puedes usar
  \emph{minhash\_generator} de \emph{textreuse}, o usar distintas
  semillas para \emph{pyhash.murmur3\_32}, o algunas de las funciones
  que generan funciones hash que vimos en clase).
\end{itemize}

\begin{enumerate}
\def\labelenumi{\arabic{enumi}.}
\setcounter{enumi}{3}
\tightlist
\item
  Utiliza el código visto en clase para encontrar pares de similitud
  alta en la colección de tweets que vimos en clase. Utiliza unos \(15\)
  hashes para encontrar tweets casi duplicados. ¿Cuántos tweets
  duplicados encontraste? ¿Qué pasa si usas menos o más funciones hash?
\end{enumerate}


\end{document}
